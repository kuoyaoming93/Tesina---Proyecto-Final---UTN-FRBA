\subsection{Tema de investigación}
Debido al creciente desarrollo de productos de RFID en la última década, se comenzó a investigar ésta tecnología con el objetivo de llevar a cabo un desarrollo nacional que pueda utilizarse en múltiples aplicaciones. Se describe en el presente documento el desarrollo de los módulos de rectificación, regulación de tensión, generador de clock, power-on-reset, modulación de carga, y demodulación de un transceptor RFID para los procesos de fabricación con el que cuenta el \textbf{Departamento de Electrónica de la Facultad Regional Buenos Aires}.
El rango de frecuencia utilizado en el proyecto es el de 13,553 a 13,567 MHz (según indica la norma ISO/IEC14443), el cual está situado en el medio del rango de longitud de onda corta del espectro electromagnético.
Se busca presentar los primeros pasos del diseño y desarrollo de un transceptor mediante la tecnología RFID. En el proceso de su desarrollo fue necesario diseñar la antena para la transmisión y recepción de información, el \textit{Front-End} analógico para la comunicación entre el lector \textit{PCD (Proximity Coupling Device)}, la tarjeta \textit{PICC (Proximity Integrated Circuit Card)} y la validación del protocolo de comunicación.

\subsection{Descripción del proyecto}

El grupo de investigación está conformado por 4 estudiantes de la carrera de Ingeniería Electrónica en la Universidad Tecnológica Nacional (Facultad Regional Buenos Aires). La investigación y desarrollo del \textit{Front-End} comenzó en el año 2015 a través de \textit{MOSIS®} en un proceso de fabricación estandar \textit{CMOS} de 500 nm (longitud mínima del canal de transistor).

El proyecto consiste en la puesta en marcha de la parte analógica de un chip transceptor RFID, el desarrollo de la parte digital del protocolo de comunicación bajo la norma ISO/IEC 14443 y por último un banco de prueba para poder verificar el correcto funcionamiento del conjunto analógico y digital con un microcontrolador que a su vez opere bajo la norma antes mencionada.

En la actualidad además del desarrollo en 500nm, también se implementó con la tecnología de 130nm para poder integrar la parte digital del proyecto al chip y además se cuenta con la posibilidad de integrar memorias.

%\subsection{Alcance del proyecto}
%El proyecto consiste en la fabricación, medición y análisis del funcionamiento del Front-End Analógico para los procesos de fabricación \textit{On Semiconductor (500 nm)} y \textit{Global Foundries (130 nm)}.
%También consiste en el armado de una placa discreta que cumpla con las mismas funciones (Modulación, demodulación, generador de clock, power on reset).
%Además se hará un banco de prueba en una \textit{FPGA (Avnet Xilinx SPARTAN 6 LX9 Microboard)} para validar el envío y recepción de los símbolos, cumpliendo la norma \textit{ISO14443-A}. La correcta interpretación de símbolos es muy importante para el funcionamiento del sistema. En lo posible, se contemplará la posibilidad de implementación del protocolo de comunicación. 